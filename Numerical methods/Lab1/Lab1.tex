\documentclass [12pt]{article}


\usepackage{ucs}
\usepackage[utf8x]{inputenc} %Поддержка UTF8
\usepackage{cmap} % Улучшенный поиск русских слов в полученном pdf-файле
\usepackage[english,russian]{babel} %Пакет для поддержки русского и английского языка
\usepackage{graphicx} %Поддержка графиков
\usepackage{float} %Поддержка float-графиков
\usepackage[left=20mm,right=15mm, top=20mm,bottom=20mm,bindingoffset=0cm]{geometry}
\usepackage{mathtools} 
\usepackage{setspace,amsmath}
\usepackage{amsmath,amssymb}
\usepackage{dsfont}
\DeclarePairedDelimiter{\abs}{\lvert}{\rvert}
\renewcommand{\baselinestretch}{1.2}
 
\usepackage{color} 
\definecolor{deepblue}{rgb}{0,0,0.5}
\definecolor{deepred}{rgb}{0.6,0,0}
\definecolor{deepgreen}{rgb}{0,0.5,0}
\definecolor{gray}{rgb}{0.5,0.5,0.5}

\DeclareFixedFont{\ttb}{T1}{txtt}{bx}{n}{12} % for bold
\DeclareFixedFont{\ttm}{T1}{txtt}{m}{n}{12}  % for normal

\usepackage{listings}

 
\lstset{
	language=Python,
	basicstyle=\ttm,
	otherkeywords={self},             % Add keywords here
	keywordstyle=\ttb\color{deepblue},
	emph={MyClass,__init__},          % Custom highlighting
	emphstyle=\ttb\color{deepred},    % Custom highlighting style
	stringstyle=\color{deepgreen},
	frame=tb,                         % Any extra options here
	showstringspaces=false            % 
}
 
\usepackage{hyperref}
 
\hypersetup{
    bookmarks=true,         % show bookmarks bar?
    unicode=false,          % non-Latin characters in Acrobat’s bookmarks
    pdftoolbar=true,        % show Acrobat’s toolbar?
    pdfmenubar=true,        % show Acrobat’s menu?
    pdffitwindow=false,     % window fit to page when opened
    pdfstartview={FitH},    % fits the width of the page to the window
    pdftitle={My title},    % title
    pdfauthor={Author},     % author
    pdfsubject={Subject},   % subject of the document
    pdfcreator={Creator},   % creator of the document
    pdfproducer={Producer}, % producer of the document
    pdfkeywords={keyword1} {key2} {key3}, % list of keywords
    pdfnewwindow=true,      % links in new PDF window
    colorlinks=true,       % false: boxed links; true: colored links
    linkcolor=black,          % color of internal links (change box color with linkbordercolor)
    citecolor=green,        % color of links to bibliography
    filecolor=magenta,      % color of file links
    urlcolor=cyan           % color of external links
}


\title{}
\date{}
\author{}

\begin{document}
\begin{titlepage}
\thispagestyle{empty}
\begin{center}
Федеральное государственное бюджетное образовательное учреждение высшего профессионального образования \\Московский государственный технический университет имени Н.Э. Баумана

\end{center}
\vfill
\centerline{\large{Лабораторная работа №1}}
\centerline{\large{по курсу <<Численные методы>>}}
\centerline{\large{<<Решение СЛАУ. Метод прогонки>>}}
\vfill
\hfill\parbox{5cm} {
           Выполнил:\\
           студент группы ИУ9-62 \hfill \\
           Иванов Георгий\hfill \medskip\\
           Проверила:\\
           Домрачева А.Б.\hfill
       }
\centerline{Москва, 2017}
\clearpage
\end{titlepage}

\textsc{\textbf{Цель:}} 

Анализ метода прогонки и решение СЛАУ с трёхдиагональной матрицей с помощью данного метода.

\textsc{\textbf{Постановка задачи:}}

\textbf{Дано:}  
$A\overline{x}=\overline{d}$, где $A \in \mathds{R}^{n \times n}$ и $ \overline{x},\overline{d} \in \mathds{R}^{n}$, $A$ - трехдиагональная матрица

\textbf{Найти:}  Решение СЛАУ с помощью метода прогонки, т.е $\overline{x}$ - $?$ при известных $A, \overline{d}$

\textbf{Тестовый пример:} 

В качестве трехдиагональной матрицы $A$  возьмем: 
$$A = \left(\begin{array}{cccc} 
4 & 1 & 0 & 0 \\
1 & 4 & 1 & 0 \\
0 & 1 & 4 & 1 \\
0 & 0 & 1 & 4  
\end{array}\right) $$
А в качестве вектора $\overline{d}$: 
$$\overline{d} = \left(\begin{array}{c} 
5 \\
6 \\
6 \\
5 \\
\end{array}\right)$$

При этом уравнение примет вид:
$$ A\overline{x}=\overline{d}
\quad   \Leftrightarrow \quad
\left(\begin{array}{cccc} 
4 & 1 & 0 & 0 \\
1 & 4 & 1 & 0 \\
0 & 1 & 4 & 1 \\
0 & 0 & 1 & 4  
\end{array}\right)
\left(\begin{array}{c} 
x_1 \\
x_2 \\
x_3 \\
x_4 \\
\end{array}\right)
 =
\left(\begin{array}{c} 
5 \\
6 \\
6 \\
5 \\
\end{array}\right) $$

\textsc{\textbf{Теоретические сведения:}}

Метод прогонки используется для решения систем линейных уравнений вида $ A\overline{x}=\overline{d}$, где $A$ - трёхдиагональная матрица. Представляет собой вариант метода последовательного исключения неизвестных. Метод прогонки был предложен И. М. Гельфандом и О. В. Локуциевским (в 1952 году; опубликовано в 1960 и 1962 годах), а также независимо другими авторами.

\textbf{Описание алгоритма:}

Пусть массив $a$ - элементы под диагональю, $b$ - на диагонали, $c$ – над диагональю.
$$\left(\begin{array}{cccccc} 
b_1 & c_1 & 0 & ... & ... &  0 \\
a_1 & b_2 & c_2 & ... & ... &  0 \\
0 & a_2 & b_3 & c_3 & ... &  0 \\
\vdots & ... & \ddots & \ddots & \ddots &  \vdots \\
0 & ... & ... & a_{n-2} & b_{n-1} & c_{n-1} \\
0 & ... & ... & ... & a_{n-1} & b_n \\
\end{array}\right) $$

\begin{equation*}
\begin{cases}
b_1x_1+c_1x_2=d_1\\
a_1x_1+b_2x_2+c_2x_3=d_2 \\
… \\
a_{n-1}x_{n-1}+b_nx_n=d_n 
\end{cases}
\end{equation*}

$$ x_1=\frac{d_1-c_1x_2}{b_1} $$
$$ a_1\frac{d_1-c_1x_2}{b_1} +b_2x_2+c_2x_3=d_2 $$
$$ x_1= \frac{d_1}{b_1} -\frac{c_1}{b_1}x_2 $$

Вводим замену переменных:
$$ \alpha_1=\frac{c_1}{b_1},\beta_1=\frac{d_1}{b_1}  $$

Проводим аналогичные действия для остальных  $x_i$, т. о. можно предположить, что решение 
$$x_{i-1}=\alpha_{i-1}x_i+\beta_{i-1},i=\overline{2,n}$$
$$x_i=\alpha_ix_{i+1}+\beta_i,i=\overline{n-1,1}$$
$$a_{i-1}x_{i-1}+b_ix_i+c_ix_{i+1}=d_i$$
$$(a_{i-1}\alpha_{i-1}+\beta_i)x_i+c_ix_{i+1}=d_i-a_{i-1}\beta_{i-1}$$
$$x_i=\frac{d_i-a_{i-1}\beta_{i-1}}{a_{i-1}\alpha_{i-1}+ b_i}-\frac{c_i}{a_{i-1}\alpha_{i-1}+b_i)x_{i+1}}$$

Вводим замену:
$$\alpha_i=-\frac{c_i}{a_{i-1}\alpha_{i-1}+b_i} $$
$$ \beta_{i} = d_i -\frac{a_{i-1}\beta_{i-1}}{a_{i-1}\alpha_{i-1}+b_i},i=\overline{2,n} $$

Получили  $\alpha_i,\beta_i$, необходимые для решения.

Вычисление $\alpha_i,\beta_i$,$i=\overline{2,n}$ называется прямым ходом метода прогонки, при этом начальные значения $\alpha_1,\beta_1$ вычисляются по формулам:
$$ \alpha_1=-\frac{c_1}{b_1}, \quad \beta_1=\frac{d_1}{b_1}  $$


Соответственно формула: 
$$a_{n-1}x_{n-1}+b_nx_n=d_n$$
$$x_n=\frac{d_n-a_{n-1}x_{n-1}}{b_n}$$
называется обратным ходом метода прогонки.

В качестве начального приближения $x_n$ выбирается значение $\beta_n$ в предположении, что $$a_{n-1}\alpha_{n-1}= 0$$

Необходимое условие метода:
$$ b_1 \neq 0 $$

Достаточные условия метода:
\begin{enumerate}
\item $\abs{b_{i}} \geq \abs{a_{i-1}} + \abs{c_{i}} $, $i=\overline{2,n-1}$ - без него возможно решение, но может и не быть.
\item $\abs{d_{i}} >\abs{c_{i}} $, $i=\overline{2,n-1}$
\end{enumerate}

\textbf{Оценка погрешности для решения СЛАУ при отсутствии точного решения:}

Найти приближенный вектор $\overline{x}^*$.

При вычислении $\widetilde{a_{ij}}$ и $\widetilde{d_i}$ с плавающей точкой возникает погрешность:
\[
  \begin{array}{l@{\quad}cr@{}l}
    && A\overline{x}^* & {}= \overline{d}^* \\
    \text{--} && A\overline{x} & {}= \overline{d} \\ \cline{2-4}
    && A(\overline{x}^*-\overline{x}) & {}= (\overline{d}^*-\overline{d})
  \end{array}
\]
или:
$$
 A\overline{\varepsilon} = \overline{r} \\
  \quad
  \Leftrightarrow
  \quad 
   \boxed{\overline{\varepsilon} = A^{-1} \overline{r}} 
$$
где $ \overline{\varepsilon} $ - вектор ошибки

Тогда исходный вектор $ \overline{x} = \overline{x}^* - \overline{\varepsilon}$.
\\

\textsc{\textbf{Практическая реализация:}}

Листинг 1. Метод прогонки для решения СЛАУ с трёхдиагональной матрицей
\begin{lstlisting}[language=python]

#!python -v
# -*- coding: utf-8 -*-

import numpy

def tridiagonalMatrixAlgorithm(d,c,a,b):
    alpha = [0]
    beta = [0]
    n = len(b)

    for i in range(1,n-1):
        if i == 1:
            alpha.append(-c[i]/d[i])
            beta.append(b[i]/d[i])
        else:
            alpha.append(-c[i]/(d[i]+alpha[i-1]*a[i-1]))
            beta.append((b[i]-a[i-1]*beta[i-1])/(d[i]+alpha[i-1]*a[i-1]))

    x = [0 for _ in range(n)]
    n -= 1
    x[n] = (b[n]-a[n-1]*beta[n-1])/(d[n]+a[n-1]*alpha[n-1])
    for i in range(n-1,0,-1):
        x[i]=x[i+1]*alpha[i]+beta[i]
    return x[1:]

def init_array(matrix,bb):
    a, b, d, c = [0], [0], [0], [0]
    d.extend([matrix[i][i] for i in range(len(matrix[0]))])
    c.extend([matrix[i][i + 1] for i in range(len(matrix[0]) - 1)])
    a.extend([matrix[i + 1][i] for i in range(len(matrix[0]) - 1)])
    b.extend(bb)
    return a,b,c,d

def valueB(matrix,x):
    b = [0 for _ in range(len(x))]
    for j in range(len(matrix)):
        b[j] = sum(matrix[j][i] * x[i] for i in range(len(matrix[j])))
    return b

def find_error(matrix,d1,d2,x2):
    r = [d2[i] - d1[i] for i in range(len(matrix))]
    matrix_rev = numpy.matrix(matrix).I
    matrix_rev = matrix_rev.tolist()

    e = [0 for _ in range(0, len(d1))]
    for j in range(0, len(matrix_rev)):
        e[j] = sum(matrix_rev[j][i] * r[i] for i in range(len(matrix_rev[j])))

    x = [x2[i] - e[i] for i in range(0, len(matrix_rev))]
    return x,e[0:4]

if __name__=="__main__":
    matrix = [[1,2,0,0],
              [2,-1,-1,0],
              [0,1,-1,1],
              [0,0,1,1]]
    D = [5,-3,3,7]
    a,b,c,d = init_array(matrix,D)
    x = tridiagonalMatrixAlgorithm(d,c,a,b)
    print(x)
    D_ = valueB(matrix,x)
    print(D_)
    x_real,error = find_error(matrix,D,D_,x)
    print(error)
    print(x_real)


\end{lstlisting}

\textsc{\textbf{Результаты:}}

Для тестирования полученной программы было выбрано в качестве трехдиагональной матрицы $A$: 
$$A = \left(\begin{array}{cccc} 
4 & 1 & 0 & 0 \\
1 & 4 & 1 & 0 \\
0 & 1 & 4 & 1 \\
0 & 0 & 1 & 4  
\end{array}\right) $$
В качестве вектора $\overline{d}$: 
$$\overline{d} = \left(\begin{array}{c} 
5 \\
6 \\
6 \\
5 \\
\end{array}\right)$$
В результате работы программы (Листинг 1) получаем значения:

$$\overline{\varepsilon} = \left(\begin{array}{c} 
0 \\
0 \\
0 \\
0 \\
\end{array}\right), \quad \overline{x} = \left(\begin{array}{c} 
1 \\
1 \\
1 \\
1 \\
\end{array}\right)$$

Как видно выше, не всегда вектор $\overline{\varepsilon}$ может содержать погрешность (нулевой вектор ошибок). Поэтому чтобы вектор ошибок не был ненулевым, протестируем нашу программу на других входных данных, где: 
$$ A = 
\left(\begin{array}{cccc} 
1 & 2 & 0 & 0 \\
2 & -1 & -1 & 0 \\
0 & 1 & -1 & 1 \\
0 & 0 & 1 & 1  
\end{array}\right), \quad \overline{d}
 =
 \left(\begin{array}{c} 
5 \\
-3 \\
3 \\
7 \\
\end{array}\right)
$$

В результате работы программы (Листинг 1) получаем значения:

$$
\overline{\varepsilon} = 
\left(\begin{array}{c} 
8.074349270001139e^{-17}\\
-4.0371746350005693e^{-17}\\
2.0185873175002846e^{-16} \\
-2.0185873175002846e^{-16} \\
\end{array}\right)
\quad
\overline{x} =
 \left(\begin{array}{c} 
0.9999999999999999 \\
2.0 \\
3.0 \\
3.9999999999999996 \\
\end{array}\right)
$$


\textbf{Выводы:}

В ходе выполнения лабораторной работы был рассмотрен метод решения СЛАУ с трёхдиагональной матрицей: метод прогонки, так же для метода была написана реализация на языке программирования Python.

Для метода прогонки можно отметить то, что отсутствует методологическая (логическая) погрешность, но присутствует вычислительная погрешность в связи с использованием чисел с плавающей запятой (нахождение обратной матрицы $A^{-1}$) , ведущая к высокому накоплению вычислительной ошибки.


\end{document}  
